% !TEX root = ../C++Prime-Plus-Notes.tex
% !TEX encoding = UTF-8

\chapter{函数重载}

\section{C++内联函数}

对于一些长度较短的函数(Google风格指南里对可以内联的函数要求为不超过十行,不包含for以及switch语句),可以进行内联处理,你可以在函数声明以及定义前加上\concept{inline}关键字来声明一个内联函数,通常的做法是省略原型,将整个定义写在函数的声明位置。内联函数以代码复制为代价,省去了函数参数压栈和弹出的时间,从而提高函数的执行效率。

有些函数即使声明为内联的也不一定会被编译器内联,比如虚函数和递归函数就不会被正常内联。通常,递归函数不应该声明成内联函数(递归调用堆栈的展开并不像循环那么简单,比如递归层数在编译时可能是未知的,大多数编译器都不支持内联递归函数)。虚函数内联的主要原因则是想把它的函数体放在类定义内,为了图个方便,抑或是当作文档描述其行为,比如精短的存取函数。而有的函数即使你没有显式地声明为内联函数如果你开了优化编译器可能也会帮你内联(编译器将内联扩展选项和关键字视为建议,不能保证所有函数都将以内联方式展开,您可以禁用内联扩展,但是不能强制编译器内联特定函数),MSVC编译器可以指定\thinspace\href{https://docs.microsoft.com/zh-cn/cpp/build/reference/ob-inline-function-expansion?view=vs-2019}{/Ob}\thinspace 参数来控制内联展开,gcc只有在开了-O优化才会处理内联。在语句层面的inline优化根本没有编译器的优化强大,所以inline关键字差不多可以被弃用了。\dpar

对于内联和宏的区别,宏是由预处理器对宏进行替代,而内联函数是通过编译器控制来实现的。而且内联函数是真正的函数,只是在需要用到的时候,内联函数像宏一样的展开,所以取消了函数的参数压栈,减少了调用的开销。你可以像调用函数一样来调用内联函数,而不必担心会产生于处理宏的一些问题,它们的代码效率是一样,但是内联函数要优于宏定义,因为内联函数遵循的类型和作用域规则,它与一般函数更相近,也有严格的参数检查。

\section{引用变量}

\subsection{创建引用变量}

对于\&符号,还有另外一个含义,就是你可以用它来声明一个引用变量,比方说\codeline{int\&\ \ }\\ \codeline{rodents = rats;},在这里rodents是rats的别名,它们的值和地址都是相同的,其实你就可以把左值引用看做是指针的语法糖,当然引用更加接近const指针,必须在创建的时候进行初始化,并且只能与一个变量绑定,不允许绑定新的变量,也就是说,你声明一个const指针也能够达到同样的效果。

\subsection{将引用作为函数参数}

对于左值引用(书上在这里都是在讨论左值引用),Google风格指南指出,所有按引用传递的参数都必须加上const,确实普通的左值引用传参在函数里可能会导致歧义,对于一个封装好的函数而言,用户可能会难以辨别哪些参数是拷贝传递哪些参数是引用传递的,所以如果你明确要修改变量的值,请用指针,虽然这可能会产生二级指针这样导致代码阅读困难的东西,但总比引起误解来的好,你如果只是想传递一个不需要修改的数组或结构,就声明一个const引用。除非你明确知道自己在干什么(标准库中使用左值引用做参数的API还是存在的),否则不要用非const的引用来作为函数参数。

\subsection{引用的属性和特别之处}

如果实参类型与引用参数不匹配时,C++将生成临时变量,对于带const的引用参数,编译器在以下两种情况生成临时变量:

\begin{enumerate}
\item 实参的类型正确,但不是左值
\item 实参的类型不正确,但可以转换为正确的类型
\end{enumerate}

\href{https://zh.cppreference.com/w/cpp/language/value_category}{左值(lvalue)}\thinspace 参数是可以被引用的数据对象,包括变量,数组元素,结构成员,引用和解除引用的指针等等,简单来说就是一个有明确存储地址的一个值,反之非左值就是右值,即一些表达式以及不能取地址的值(在C语言中的左值指的就是可以出现在表达式左边的值,但现在const变量也被视为左值,虽然它是一个不可修改的左值)。

如果实参的类型不匹配造成的类型转换,编译器将创建类型正确的临时变量,但如果这里的参数类型不带const,编译器将会报错[a reference of type \fira{"}int \&\fira{"} (not const-qualified) cannot be initialized with a value of type \fira{"}long\fira{"}],只有对于const类型的引用,编译器才会允许进行正确的类型转换。

\subsection{将引用用于结构}

这里讨论有关返回引用的问题,你可以把函数的返回值设为引用,但你应该避免返回的引用对象在函数运行完就不复存在的情况,因此最好返回函数调用前就已经存在的对象的引用,或者你也可以返回在函数中new的堆空间,但这样可能会导致在函数运行完毕后忘记释放空间,从而导致内存泄漏。普通的返回会将函数的返回值先复制到一个临时位置,让后再做操作,而返回引用仅仅是挪一挪指针而已,这样的操作效率会搞得多。

\section{默认参数}

你可以在声明函数原型时对参数添加一些默认值,这样在函数没有接收到某些参数时仍能通过默认值运行。但是在Google风格指南指出:缺省参数会干扰函数指针,导致函数签名(function signature)与调用点的签名不一致,即在一个现有函数添加缺省参数,就会改变它的类型,那么调用其地址的代码可能会出错。此外,缺省参数会造成臃肿的代码,毕竟它们在每一个调用点都有重复(调用函数的代码表面上看来省去了不少参数,但编译器在编译时还是会在每一个调用代码里统统补上所有默认实参信息,造成大量的重复)。因此在需要用到默认参数的地方,建议你都用函数重载,因为函数重载均没有这些问题。

这里提到了一个函数签名的概念,函数签名包含了一个函数的信息,包括函数名,参数类型,所在的类和名称空间及其他信息,函数签名用于识别不同的函数。注意区别这个概念和名称修饰的区别,名称修饰是在编译器及链接器处理符号时,使用某种名称修饰的方法,使得每个函数签名对应一个修饰后名称(decorated name)。编译器在将C++源代码编译成目标文件时,会将函数和变量的名字进行修饰,形成符号名,也就是说,C++的源代码编译后的目标文件中所使用的符号名是相应的函数和变量的修饰后名称。C++编译器和链接器都使用符号来识别和处理函数和变量,所以对于不同函数签名的函数,即使函数名相同,编译器和链接器都认为它们是不同的函数。

\section{函数重载}

\subsection{重载示例}

函数重载又叫做函数多态,它可以利用函数的参数列表,也叫做\concept{函数特征标(function signature)}来区分同一个函数名称的不同版本,首先函数特征标不区分是否为引用类型,因为对于函数调用而言,编译器无法区分你传入的参数是需要拷贝传递还是引用传递,其次在匹配函数时,并不区分函数的返回值,也就是说你不能够通过返回值来重载函数,但不同特征标的函数可以有不同的返回值

类设计和STL经常使用引用参数,其中的重载的类型匹配有三种:左值引用参数与可修改的左值参数能够匹配,const左值引用参数能够与可修改的左值,const左值和右值参数匹配,而右值引用参数只能与右值相匹配,在这里const左值引用能够与三者都匹配,如果重载使用这三种参数的函数,编译器将选择最匹配的版本来调用。

\subsection{何时使用函数重载}

书上在这里的说法是能够用默认参数解决的问题就不要上函数重载了,这里应该面向一些小的项目,函数重载还会多浪费内存,而上面的Google风格指南应该是面向大的project,建议你用函数重载。

\section{函数模板}

\addtocounter{subsection}{2}

\subsection{显式具体化}

对于模板的显式具体化,C++98标准用以下的规则:

\begin{enumerate}
\item 对于给定的函数名,可以有非模板函数,模板函数和显式具体化模板函数以及它们的重载版本。
\item 显式具体化的原型和定义应以template<>打头,并通过名称来指出类型。
\item 具体化优先于常规模板,而非模板函数优先于具体化和常规模板。
\end{enumerate}

\subsection{实例化和具体化}

\concept{实例化(instantiation)},这个词的意思就像你从类里面实例化出一个对象一样,当编译器使用模板为特定类型生成函数定义的过程就是在实例化,从一个高度抽象的模板中生成一个具体的函数,也就是模板实例。最初的编译器只支持\concept{隐式实例化(implicit instantiation)},也就是你只能让编译器去选择该怎么来实例化一个模板,但现在的C++还可以\concept{显式实例化(explicit instantiation)}一个模板,这意味着你可以直接命令编译器来创建特定类型的模板实例,即\codeline{template void fun<int>(int);},当然你也可以简写,<int>可以写成<>甚至去掉,编译器会推导出推导出模板实参。书上把这句话放在了main()函数内,但是这样似乎做会报错,Intellisence会警告你explicit instantiation is not allowed in the current scope,所以应该是只能在函数外进行显式实例化。具体显式实例化有什么用,好像是没什么用的(应该是我水平不够,看不出来它有什么用)。当然你也可以在程序中使用函数来创建显式实例化,比如说这样去调用函数\codeline{fun<int>(1)},如果你这里的实例化类型与参数类型不匹配,编译器就会将其转换为正确的类型,如果这里不能正确转换,编译器会报错。

这里还有一个概念就是\concept{显式具体化(explicit specialization)},与显式实例化不同,显式具体化的函数需要有自己的函数声明,因为它们本来就是为了解决某些模板解决不了的类型。你可以这样声明\codeline{template <> void fun<int>(int);},当然你也同样可以简写,<int>可以写成<>或者没有。如果你试图在同一个文件中将同一个类型显式实例化和显式具体化,编译器将报错。

\subsection{编译器选择使用哪个函数版本}

对于函数重载,函数模板,函数模板重载导致有多个备选函数时,编译器会进行一个叫\concept{重载解析(overloading resolution)}的过程,其步骤如下:

\begin{enumerate}
\item 创建候选函数列表,其中包含与被调用函数名称相同的重载函数和模板函数。
\item 使用候选函数列表创建可行函数列表。这些都是参数数目正确的函数,为此有一个隐式转换序列,其中包括实参类型与相应形参类型完全匹配的情况。
\item 确定是否有最佳的可行函数。如果有则使用,否则将调用出错。
\end{enumerate}

在确定最佳函数时,编译器会根据一下顺序来挑选函数:

\begin{enumerate}
\item 完全匹配,但允许一些无关紧要的转换(见\tref{table:Match}),但常规函数优先于模板
\item 提升转换,例如char和short转换为int,float转换为double
\item 标准转换,例如int转换为char,long转换为double
\item 用户定义的转换,例如类声明的转换
\end{enumerate}

\begin{table}[!hbt]
\centering
\begin{tabular}{p{18em}|p{18em}}
\hline
\stress{从实参} & \stress{到形参} \\
\hline
Type & Type\& \\
\hline
Type\& & Type \\
\hline
Type[] & \fira{*}Type \\
\hline
Type(argument-list) & Type(\fira{*})(argument-list)\\
\hline
Type & const Type \\
\hline
Type & volatile Type \\
\hline
Type\fira{*} & const Type\fira{*} \\
\hline
Type\fira{*} & volatile Type\fira{*}\\
\hline
\end{tabular}
\caption{完全匹配允许的无关紧要的转换}
\label{table:Match}
\end{table}

如果这些函数都是完全匹配,则选择顺序为:非模板函数>显式具体化||显式具体化>隐式实例化。如果两个完全匹配的函数都是隐式实例化模板函数,则选择最具体(most spe-\\ cialized)的模板函数,这里的最具体指的是编译器推断使用哪种类型时执行的转换最少。用于找出最具体的模板的规则叫做函数模板的部分排序规则(partial ordering rules),这一部分比较复杂,参见\thinspace\href{https://zh.cppreference.com/w/cpp/language/overload_resolution}{cppreference[重载决议]}\thinspace。

\subsection{模板函数的发展}

C++11新增了一个\thinspace\href{https://zh.cppreference.com/w/cpp/language/decltype}{\concept{decltype}}\thinspace 关键字,你可以用一个表达式来声明变量的类型,这在编写模板时非常好用,因为有时你要声明的类型可能暂时无法确定,这时就可以用表达式来代替,即\codeline{decltype(expression) var;}。编译器在确定var的类型时会遍历一个核对表,其简化版如下:

\begin{enumerate}
\item 如果expression是一个没有被括号括起的标识符,则var的类型与改标识符的类型相同,包括const。
\item 如果expression是一个函数调用,则var的类型与函数的返回类型相同(注意这里并不会真的去调用函数,编译器只是回去查看函数的原型来确定返回类型)。
\item 如果expression是一个被括号括起的左值,则var为指向其类型的引用。
\item 如果前面的条件均不满足,则var与expression的类型相同。
\end{enumerate}

有时,函数的返回值类型也无法确定,为此C++11加入了\concept{后置返回类型(trailing return type)},你可以这样来声明一个模板:\codeline{template <typename T, typename U> auto\ \ }\\ \codeline{add(T x, U y) -> decltype(x + y);},但是Coogle风格指南指出这样的写法可能让某些读者陌生,它建议采用以前的std::declval()函数,其声明为:\codeline{template <typename\ \ }\\ \codeline{T, typename U> decltype(std::declval<T\&>() + std::declval<U\&>())\mbox{\ \hspace{-0.316em}\ }}\\ \codeline{add(T x, U y);},但是对于Lambda表达式来说,后置返回类型是显式指定Lambda表达式返回值的唯一方式。

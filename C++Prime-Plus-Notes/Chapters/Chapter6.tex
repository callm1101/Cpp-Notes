% !TEX root = ../C++Prime-Plus-Notes.tex
% !TeX encoding = UTF-8

\chapter{分支语句和逻辑运算符}

\addtocounter{section}{1}

\section{逻辑表达式}

\addtocounter{subsection}{2}

\subsection{用\&\&来设置取值范围}

如果你想判断一个值大于一个数并且小于一个数,不要把代码写成\codeline{0 < score < 100}\\ (当然如果是python你可以这么做也最好怎么做),因为编译器通过从左向右的结合法则会把上面的代码当成\codeline{(0 < score) < 100}来执行,也就是说不管如何前面表达式的值要么是0,要么是1,都会小于100,所以正确的方法应该是\codeline{socre > 0 \&\& score < 100}。

\addtocounter{subsection}{2}

\subsection{逻辑运算符细节}

对于复合比较的语句,\&\&运算符的优先级要高于||的优先级,并且C++确保程序从左向右计算逻辑表达式,并且在知道答案后立即停止,也就是说你可以写出\codeline{x != 0 \&\& 1.0 /}\\ \codeline{x > 100}这样的代码而不用担心编译器会报错。

\addtocounter{section}{1}

\section{?:运算符}

\concept{条件运算符(?:)},又叫三元运算符,该运算符的通用格式为expression1 ? expression2 : expression3,当expression1为真时,表达式返回expression2的值,反之返回express-\\ ion3的值。

\addtocounter{section}{1}

\section{break和continue语句}

break和continue语句都能使程序跳过当前循环中一部分代码,可以在switch语句中使用break语句来跳过剩余的case以及default,当\concept{break}用于循环中时,它会跳过循环的剩余部分去执行下面的语句,而\concept{continue}用于循环中时会跳过循环体的剩余部分,而开始新一轮的循环。当然,如果你偏爱goto,它也能完成工作,只是会降低代码的可读性罢了。

\section{读取数字的循环}
\label{section:cinscore}

现在你接到了一个来自甲方的任务,是要写一个能够录入分数的程序,但你面向的客户也许根本不知道什么是键盘,所以你要保证你的程序不会有任何bug(因为书上在这段程序中用到了continue,所以在这里才讲到,但我感觉书上的程序还是有bug,所以我改写了一下)。

\begin{cpp}
#include <iostream>
#include <limits>

int main(int argc, char|*| argv[]) {
    short score;
    std::cin.sync();
    while (!(std::cin >> score) !!||!! (score < 0 !!||!! score > 100) !!||!!
            std::cin.get() != '\n') {
        std::cin.clear();
        std::cin.ignore(std::numeric_limits<std::streamsize>::
            max(), '\n');
        std::cout << "Invalid input!" << std::endl;
    }
    return 0;
}
\end{cpp}

这里首先使用std::cin.sync()函数来清空cin缓冲区里面未读取的信息,接下来的!(std::\\ cin >> score)对应输入非法字符(包括检测到EOF)的情况,(score < 0 || score > 100)对应当你输入的分数不合法的情况,而std::cin.get != \leftqm\mybackslash n\rightqm 则会检查当你输入类似于像85c这样的字符时cin会帮你自动截断当成85的情况,一旦当你输入的类型有问题时,首先使用std::\\ cin.clear()来重置EOF改变的标志位,接着用std::cin.ignore(std::numeric\_limits<std::\\ streamsize>::max(), \leftqm\mybackslash n\rightqm);来清空缓冲区剩余的字符(这里使用numeric\_limits宏来获取输入流中字符数量上限,将\leftqm\mybackslash n\rightqm 作为delim告知函数清除到换行符),最后告知用户Invalid input!

\section{简单文件输入/输出}

\addtocounter{subsection}{2}

\subsection{读取文本文件}

当你用你定义的std::ifstream对象(暂且就叫它fcin)来打开文件时,可以使用\codeline{fcin.}\\ \codeline{is\_open()}来判断文件是否被正常地打开,如果文件被成功打开,方法is\_open()会返回true,反之则会返回false,如果你想让程序在打开文件失败后中止,你可以使用\concept{cstdlib}中定义地exit()函数,在该头文件中,还定义了一个用于通操作系统通信的宏EXIT\_FALIURE,也就是说,你可以使用\codeline{exit(EXIT\_FALIURE)}来终止程序。

\begin{cpp}
#include <iostream>
#include <fstream>
#include <cstdlib>

int main(int argc, char|*| argv[]) {
    char ch;
    std::ifstream fcin;
    fcin.open("file");
    if (fcin.is_open() !!&&!! fcin.peek() != EOF) {
        while ((ch = fcin.get()) != EOF) {
            // do something
        }
    }
    else {
        exit(EXIT_FAILURE);
    }
    fcin.close();
    return 0;
}
\end{cpp}

在上面的程序中用到的peek()函数能够读取并返回下一个字符,但并不提取该字符到输入流中,也就是说你可以用这个函数来窥探输入流中的下一个字符,但并不将它提出。这里使用peek()函数来判断EOF而不用eof()函数是因为eof()返回true的条件是读到\concept{文件结束符(0xf\/f)},而不是文件内容的最后一个字符,也就意味着用eof()来判断总会迟滞一位。如果你在这里的if判断用\codeline{fcin.eof()},你会发现它无法判断出空文件,因为当你打开一个文件时,文件指针位置为0,并不是指向第1个字符,所以这里用的peek()会返回输入流中可用的下一个字符(就是文件指针指向的下一位),对于空文件来说就是EOF。

如果你在这里的while判断用的是\codeline{!fcin.eof()},而把fcin.get(ch)写在循环里,你会发现文件的最后一个字符被读取了两次,就是因为当输入流中的EOF被读入时并不会改变ch的值,而是将eof\/bit置为1。

但如果你想输入的不是char型的话,peek()会表现很差,具体会在后面章节详细说明。

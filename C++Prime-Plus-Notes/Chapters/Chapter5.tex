% !TEX root = ../C++Prime-Plus-Notes.tex
% !TeX encoding = UTF-8

\chapter{循环和关系表达式}

\addtocounter{section}{3}

\section{基于范围的for循环(C++11)}

C++11新增了一种基于\stress{范围(ranged-based)}的for循环,这简化了对于数组或容器类(对于容器可能用迭代器会更好)的循环,即:

\begin{cpp}
#include <iostream>
#include <vector>

int main(int argc, char|*| argv[]) {
    int arr[] = {1, 2, 3};
    std::vector<std::string> vec = {"abc", "def"};
    for (int i : arr) {
        std::cout << i << std::endl; // 1 2 3
    }
    for (std::vector<std::string>::iterator itor = vec.begin();
            itor != vec.end(); itor++) {
        std::cout << |*|itor << std::endl; // abc def
    }
    return 0;
}
\end{cpp}

\section{循环和文本输入}

\subsection{使用原始的cin进行输入}

这里底下都是要解决循环读取来自键盘的文本输入,并以一个你喜欢的方式结束读取的问题。你首先肯定想到利用while循环和某一个特殊字符来完成这个工作,即:

\begin{cpp}
#include <iostream>

int main(int argc, char|*| argv[]) {
    char ch;
    std::cin >> ch;
    while (ch != '#') {
        std::cin >> ch;
    }
    return 0;
}
\end{cpp}

上面的这个程序确实能够完成不断从键盘中读入文本,并以\#结束的工作,但你很容易发现,你没办法输入空格或者换行符,因为这些字符都会被cin过滤,你输入的信息会先送入缓冲区,当你按下回车,这些字符才会被发送给cin,这也就是你甚至能够在\#后面输入字符,虽然这并没有什么用。

\subsection{使用cin.get(char)进行补救}

如果你想逐个字符地检查,\codeline{std::cin.get(ch)}可以帮你实现这一点,它会读取输入流中的下一个字符,即使它是空格,即:

\begin{cpp}
#include <iostream>

int main(int argc, char|*| argv[]) {
    char ch;
    std::cin.get(ch);
    while (ch != '#') {
        std::cin.get(ch);
    }
    return 0;
}
\end{cpp}

\subsection{使用哪一个cin.get()}

对于std::cin.get()这个函数到现在应该已经出现过了3个版本,总共有4个重载的版本:

\begin{enumerate}
\item int cin.get()
\item istream\& cin.get(char\& var)
\item istream\& cin.get(char\fira{*} s, streamsize n)
\item istream\& cin.get(char\fira{*} s, streamsize n, char delim)。
\end{enumerate}

它们的作用我们前面已经分别讲过,其中streamsize在Linux下被定义为long int型,最后那个delim参数为中止字符,具体你可以看istream.tcc\footnote{linux下一般在\codeline{/usr/include/C+\/+/[0\~{}9]/istream.tcc}}这个文件里面那些好长的模板类函数。后面三个类型的返回值都是cin对象,也就是说,你能够写出像\codeline{std::cin.get(}\\ \codeline{ch1).get(ch2)}这种代码,它将两个字符连续地读入ch1和ch2。

\subsection{文件尾条件}

使用\#来结束输入总很难让人满意,因为你有的时候可能回想输入\#(其实就是不够优雅),这个时候你可以尝试使用\concept{检测文件尾(EOF)}的方法来告知程序结束输入。EOF是end of f\/ile的缩写,Wiki上对其的定义为a condition in a computer operating system where no more data can be read from a data source,就是文件尾的一个特殊标识符,当然你也可以通过键盘来模拟EOF,在Linux下是Ctrl + D,在Windows下是Ctrl + Z和Enter。在检测到EOF后,cin会将\concept{eof\/bit}和\concept{failbit}都设置为1,你可以通过\codeline{std::cin.eof()}和\codeline{std::cin.fail()}两个函数来监测是否达到了文件尾,如果eof-bit被设置为1,则eof()会返回bool值true,如果eof\/bit或failbit被设置为1,则fail()均会返回bool值true,使用fail()会比eof()更广泛,因为failbit还能监控到其他故障(磁盘故障)。

当cin检测到EOF后,会设置eof\/bit和failbit,此后cin将会停止读取输入,再次调用也不管用(对于文件输入,这有助于程序读取不会超出文件尾)。但对于键盘输入,你可能后面想要再次调用cin,这时可以用\codeline{std::cin.clear()}来清除EOF标记,但在有些系统中,可能Ctrl + Z就会结束输入输出,用clear()也无法恢复。

同时,由于std::cin.get(char)的返回值为cin,你可以将这个直接作为循环的条件,也就是说上面的while语句可以改写成\codeline{while (std::cin.get(ch))},而且这样可以去掉while前和while内的get()语句。

\subsection{另一个cin.get()版本}

为了使用syd::cin.get(),需要知道它对于EOF的处理。当该函数到达EOF时,将返回一个用符号常量EOF表示的特殊值,通常这个值为-1,因为没有ASCII码为-1的字符,也就不会与任何输入的字符弄混,所以你只需要在程序中使用EOF即可。对于\codeline{std::cin.get()}\\ 和\codeline{std::cin.get(ch)}的区别,见\tref{table:cinget diff}。

\begin{table}[!htb]
\centering
\begin{tabular}{p{12em}|p{12em}|p{12em}}
\hline
\stress{属性} & \stress{cin.get(ch)} & \stress{ch = cin.get()} \\
\hline
传递输入字符的方式 & 赋给参数ch & 将函数返回值赋给ch \\
\hline
用于字符输入时函数的返回值 & istream对象(执行bool转换后为true) & int型的字符编码 \\
\hline
到达EOF时函数的返回值 & istream对象(执行bool转换后为true) & EOF \\
\hline
\end{tabular}
\caption{cin.get(ch)与cin.get()}
\label{table:cinget diff}
\end{table}

有关能够让你完全正确输入的程序(针对那种不想好好的用你的程序而在键盘上狂敲EOF的用户),会在\secref{section:cinscore}给出。当然,你也可以用std::cin.get()函数来重写上面的程序,即:

\begin{cpp}
#include <iostream>

int main(int argc, char|*| argv[]) {
    char ch;
    while ((ch = cin.get()) != EOF) {
        // do something
    }
    return 0;
}
\end{cpp}

对于上面程序的while判断中的\codeline{(ch = cin.get()) != EOF},由于赋值语句的值就是左操作数的值,所以这里ch = cin.get()返回的就是ch的值,这也是这里为何可以进行判断的原因。

\section{嵌套循环和二位数组}

假设你现在有一个二位数组\codeline{int arr[4][5]},你想访问里面的第3排第4列的那个元素,现在你有以下四种办法(排名不分先后):

\begin{itemize}
\item arr[2][3]
\item \fira{*}(arr + 2)[3]
\item \fira{*}(\fira{*}(arr + 2) + 3)
\item \fira{*}(\fira{*}arr + 13)
\end{itemize}

也就是说,对于一个二位数组,其实它的本质是一个有行数个元素的一维指针数组,其中每一个指针指向每行的有列数个元素的一位数组的第一个元素,所以对于二位数组名,它应该是一个指向一个指针数组的指针。

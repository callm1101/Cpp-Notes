% !TEX root = ./C++Prime-Plus-Notes.tex
% !TeX encoding = UTF-8

% 设置字号为五号,行距为1.5倍
\documentclass[UTF8,a4paper,zihao=5,linespread=1.5]{ctexrep}

% 定制页面版式的宏包
\usepackage[a4paper,left=3cm,right=3cm, bottom=3cm,top=3cm]{geometry}

% 设置字体为思源宋体
\setmainfont{Source Han Serif SC}
\setCJKmainfont{Source Han Serif SC}

% 设置无衬线体为思源黑体
\setsansfont{Source Han Sans SC}
\setCJKsansfont{Source Han Sans SC}

% 数学环境中使用楷体
\setCJKmathfont{KaiTi}

% 定义一些其他字体
\newfontfamily{\firacode}{Fira Code}
\newfontfamily{\consolas}{Consolas}
\newfontfamily{\calibriitalic}{Calibri Italic}

% 只改变CJK字体
\setCJKfamilyfont{shheiti}{Source Han Sans SC}
\newcommand{\shheiti}{\CJKfamily{shheiti}}

% 定制日期时间格式的宏包
\usepackage[yyyymmdd]{datetime}
\renewcommand{\dateseparator}{.}

% 设置整篇的行间距的最小值和最大值
\setlength{\lineskiplimit}{3pt}
\setlength{\lineskip}{3pt}

% 强调和概念
\newcommand{\stress}[1]{{\bfseries{#1}}}
\newcommand{\concept}[1]{{\bfseries{#1}}}

% 下划线和删除线宏包
\usepackage[normalem]{ulem}

% 自己定义的半角引号
\newcommand{\leftqm}{\makebox[0.5em][r]{`}}
\newcommand{\rightqm}{\makebox[0.5em][l]{'}}

% 空两行
\newcommand{\dpar}{\\ \mbox{}}

% 一些其他的小定义
\newcommand{\fira}[1]{{\firacode{#1}}}
\newcommand{\mybackslash}{{\char92}}

% 目录和章节样式
\usepackage{titlesec}
\usepackage{titletoc}

% 设置目录样式
\titlecontents{chapter}[3em]{\bfseries\normalsize}{\contentslabel{4.5em}}{}{\hfill\contentspage}[\vspace{0.5ex}]
\titlecontents{section}[3em]{\normalsize}{\contentslabel{3em}}{}{\titlerule*[8pt]{$\cdot$}\contentspage}[\vspace{0.5ex}]
\titlecontents{subsection}[7em]{\normalsize}{\contentslabel{4em}}{}{\titlerule*[8pt]{$\cdot$}\contentspage}[\vspace{0.5ex}]

% 控制标题排序深度到subsection
\setcounter{secnumdepth}{3}

% 设置标题样式
\titleformat{\chapter}[display]{\centering\LARGE\bfseries}{}{0.5em}{}
\titleformat{\section}[hang]{\centering\Large\bfseries}{\bfseries\Large \S\ \thesection}{0.5em}{}
\titleformat{\subsection}[hang]{\large\bfseries}{\bfseries\large \S\ \thesubsection}{0.5em}{}

% 编号列表宏包,并自定义了三个列表
\usepackage[inline]{enumitem}
\setlist[enumerate]{font=\bfseries,itemsep=0pt}
\setlist[itemize]{font=\bfseries,leftmargin=\parindent}
\setlist[description]{font=\bfseries\uline,labelindent=\parindent,itemsep=0pt,parsep=0pt,topsep=0pt,partopsep=0pt}

% 设置图表格式
\usepackage{array,multirow,makecell,tabularx}

% 设置图表跨页
\usepackage{longtable}

% 行高增加
\setlength\extrarowheight{2pt}

% 表格行距改为1.2倍
\renewcommand\arraystretch{1.2}

% 图片文字混排宏包
\usepackage{graphicx,wrapfig}

% 设置不需要浮动体的表格和图像标题
\setlength{\abovecaptionskip}{7pt}
\setlength{\belowcaptionskip}{1pt}

% 引用名称定义
\newcommand{\charef}[1]{第\ref{#1}章}
\newcommand{\secref}[1]{\S\ref{#1}\ }
\newcommand{\subsecref}[1]{\S\ref{#1}\ }
\newcommand{\pref}[1]{第\pageref{#1}页}
\newcommand{\fref}[1]{图\ref{#1}}
\newcommand{\tref}[1]{表\ref{#1}}

% 重定义图表名称
\renewcommand{\tablename}{表}
\renewcommand{\figurename}{图}

% 数学必备宏包
\usepackage{amsmath}
\usepackage{amssymb}

% 定理环境
\usepackage{amsthm}
\theoremstyle{definition}\newtheorem{laws}{Law}[section]
\theoremstyle{plain}\newtheorem{ju}[laws]{Jury}
\theoremstyle{remark}\newtheorem*{marg}{Margaret}

% 使用线性代数字体
\usepackage{mathrsfs}

% 定义可能使用到的颜色
\usepackage{xcolor}
\definecolor{keywordscolor}{RGB}{94,94,255}

% 复制代码时不复制行号
\usepackage{accsupp}
\newcommand{\emptyaccsupp}[1]{\BeginAccSupp{ActualText={}}#1\EndAccSupp{}}

% 设置代码环境的盒子
\usepackage{tcolorbox}
\tcbuselibrary{listings,skins,breakable}

% 设置代码环境样式
\usepackage{listings}
\lstset{
    columns=spaceflexible,                                  % 设置行内单词间距的处理方式
    frameround=tttt,                                        % 设置代码框为圆角
    basicstyle=\firacode\shheiti,                           % 设置字体
    numbers=left,                                           % 在左侧显示行号
    numberstyle=\emptyaccsupp,                              % 复制代码不复制行号
    frame=none,                                             % 不显示背景边框
    backgroundcolor=\color{gray!10!white},                  % 设定背景颜色,与下面相同
    keywordstyle=\color[RGB]{94,94,255},                    % 设定关键字颜色
    numberstyle=\footnotesize\color[RGB]{107,107,107},      % 设定行号格式
    commentstyle=\calibriitalic\kaishu\color[RGB]{0,96,96}, % 设置代码注释的格式
    stringstyle=\color[RGB]{128,0,0},                       % 设置字符串格式
    showstringspaces=false,                                 % 不显示字符串中的空格
    emphstyle=\color[RGB]{112,64,160},                      % 设置强调词的颜色
    language=c++,                                           % 设置语言为C++
    moredelim=[is][\color{keywordscolor}\consolas]{|}{|},   % firacode中*有bug,改用consolas
    moredelim=[is][\color{keywordscolor}\firacode]{!!}{!!}, % 手动设置&高亮
    morekeywords={alignas,continue,friend,register,true,alignof,decltype,goto,
    reinterpret_cast,try,asm,defult,if,return,typedef,auto,delete,inline,short,C++
    typeid,bool,do,int,signed,typename,break,double,long,sizeof,union,case,
    dynamic_cast,mutable,static,unsigned,catch,else,namespace,static_assert,using,
    char,enum,new,static_cast,virtual,char16_t,char32_t,explict,noexcept,struct,
    void,export,nullptr,switch,volatile,class,extern,operator,template,wchar_t,
    const,false,private,this,while,constexpr,float,protected,thread_local,
    const_cast,for,public,throw,std},
    emph={map,set,multimap,multiset,unordered_map,unordered_set,
    unordered_multiset,unordered_multimap,vector,string,list,deque,
    array,stack,forwared_list,iostream,fstream,cstdlib,memory,shared_ptr,unique_ptr,random,
    bitset,ostream,istream,cout,cin,wcout,wcin,endl,climits,limits,move,default_random_engine,
    uniform_int_distribution,iterator,begin,end,algorithm,functional,bing,numeric}
}

% 设置行间代码环境
\newtcbox{\codeline}[1][cyan]{
    on line,before upper=\firacode\shheiti,arc=0pt,outer arc=0pt,colback=#1!10!white,colframe=#1!50!black,
    boxsep=0pt,left=1pt,right=1pt,top=1pt,bottom=1pt,boxrule=0pt,bottomrule=1pt,toprule=1pt
}

% 设置C++代码环境的盒子
\newtcblisting{cpp}{
    breakable,skin=bicolor,colback=gray!10!white,colbacklower=white,colframe=white!50!black,listing only,
    left=6mm,top=2pt,bottom=2pt,fontupper=\small,listing options={}
}

% 设置Python代码环境的盒子
\newtcblisting{python}{
    breakable,skin=bicolor,colback=gray!10!white,colbacklower=white,colframe=white!50!black,listing only,
    left=6mm,top=2pt,bottom=2pt,fontupper=\small,listing options={language=python}
}

% 设置参考文献的格式
\bibliographystyle{plain}

% 将引用添加到目录
\usepackage[numindex,notbib]{tocbibind}
\usepackage[square,super,sort&compress]{natbib}

% 设置引用
\usepackage{hyperref}
\hypersetup{
    colorlinks,linkcolor=black,anchorcolor=black,citecolor=teal,urlcolor=teal,
    bookmarksnumbered=true,pdftitle=C++ Prime Plus 学习笔记,pdfauthor=Calm Liu,pdfstartview=FitH
}
